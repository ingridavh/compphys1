\documentclass[11pt]{article}
\usepackage{graphicx}
\usepackage[utf8]{inputenc} 
\usepackage{epstopdf}
\usepackage[makeroom]{cancel}
\usepackage{amsmath}

\begin{document}
\title{FYS4150: Project 1}
\author{Ingrid A V Holm}
\maketitle

\section*{a)}
\begin{flushleft}

We want to solve the one-dimensional Poisson equation with Dirichlet boundary conditions:

\begin{equation}
-u''(x) = f(x), \textbf{ } x \in (0,1), \textbf{ } u(0) = u(1) = 0
\end{equation}

We approximate the second derivative using the 3 point formula:

\begin{equation}
- \frac{v_{i+1} + v_{i-1} - 2 v_i}{h^2} = f_i 
\end{equation}

For $i = 1,2,...,n$, where $f_i = f(x_i)$. We rewrite this equation:

$$
- v_{i-1} + 2 v_i - v_{i+1}= h^2 f_i 
$$

Setting in $i = 1,2,3$ we start to see a pattern:

$$
- v_0 + 2 v_1 - v_2= h^2 f_1 
$$

Since $v_0 = 0$:

$$
2 v_1 - v_2= h^2 f_1 
$$

$$
- v_1 + 2 v_2 - v_3= h^2 f_2 
$$

$$
- v_2 + 2 v_3 - v_4= h^2 f_3 
$$

We can write this as a matrix working on a vector:

$$
\begin{pmatrix}
2 & -1 & 0 &...& ... & 0\\
-1 & 2 & -1 & 0 & ... & ...\\
0 & -1 & 2 & -1 & 2 & ...\\
& ... & ... & ... & ... & ...\\
0 & ... & & -1 & 2 & -1\\
0 & ... & & 0 & -1 & 2\\
\end{pmatrix}
\begin{pmatrix}
v_1\\
v_2\\
...\\
...\\
...\\
v_n\\
\end{pmatrix}
=
\begin{pmatrix}
\tilde{b}_1\\
...\\
...\\
...\\
\tilde{b}_n\\
\end{pmatrix}
$$

$$
\textbf{A} \textbf{v} = \textbf{b}
$$


\end{flushleft}

\begin{flushleft}
We assume a source term and closed-form solution:

\begin{equation}
f(x) = 100e^{-10x}
\end{equation}

\begin{equation}
u(x) = 1 - (1-e^{-10})x - e^{-10x}
\end{equation}

Check that this satisfies the Poisson equation:

$$
u'(x) = - 1 + e^{-10} + 10 e^{-10x}
$$

$$
u''(x) = -100 e^{-10x} = - f(x)
$$


\end{flushleft}


\subsection*{b)}

\begin{flushleft}
We rewrite the set of equations in terms of a tridiagonal matrix:

$$
\begin{pmatrix}
b_1 & c_1 & 0 & ... & ... & ...\\
a_2 & b_2 & c_2 & ... & ...\\
0 & a_3 & b_3 & c_3 & ... & ...\\
& ... & ... & ... & ...& ...\\
& & & a_{n-2} & b_{n-1} & c_{n-1}\\
& & & & a_n & b_n\\ 
\end{pmatrix}
\begin{pmatrix}
v_1\\
v_2\\
...\\
...\\
...\\
v_n\\
\end{pmatrix}
=
\begin{pmatrix}
\tilde{b}_1\\
\tilde{b}_2\\
...\\
...\\
...\\
\tilde{b}_n\\
\end{pmatrix}
$$

In index notation this becomes:

\begin{equation}
a_i v_{i-1} + b_i v_i + c_1 v_{i+1} = \tilde{b}_i
\end{equation}

With $a_1= c_n = 0$. The algorithm for solving this set of equations consists of two steps; a decomposition and forward substitution followed by a backward substitution:
\end{flushleft}

\begin{flushleft}

Step 1, forward substitution:
\begin{equation*}
k
\end{equation*}


Step 1:

\begin{flalign*}
&\beta = b_1 &\\
&\gamma = vector[n]&\\
&v_1 = \frac{\tilde{b}_1}{b_1}&\\
&\text{for }(j=2; j \leq n; j = j+1):&\\
& \text{ } \text{ }  \text{ }\text{ } \gamma_j = \frac{c_{j-1}}{\beta} &\\
&\text{ }\text{ }  \text{ }\text{ }\beta = b_j-a_j*\gamma_j&\\
& \text{ } \text{ } \text{ }\text{ } v_j = \frac{(\tilde{b}_j-a_j \cdot v_{j-1})}{\beta}&\\
\end{flalign*}


Step 2:

\begin{flalign*}
&\text{for } (j=n-1; j \geq 1; j = j-1):&\\
& \text{ } \text{ } \text{ }\text{ } v_j = \gamma_{j+1} \cdot v_{j+1}&\\
\end{flalign*}

\end{flushleft}




\section*{References}

\begin{flushleft}
Use BibTex!
\end{flushleft}















\end{document}