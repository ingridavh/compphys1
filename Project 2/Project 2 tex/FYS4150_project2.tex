\documentclass[11pt]{article}
\usepackage{graphicx}
\usepackage[utf8]{inputenc} 
\usepackage{epstopdf}
\usepackage[makeroom]{cancel}
\usepackage{framed}
\usepackage{cite}
\usepackage{hyperref}
\usepackage{amsmath}
\usepackage{amsfonts}
\usepackage{bbold}

\begin{document}
\title{FYS4150: Project 2}
\author{Ingrid A V Holm}
\maketitle

\subsection*{Unitary and orthogonal transformations (a)}

\begin{flushleft}
A unitary matrix $U$ has the properties: 

\begin{equation}
U^{-1} = U^{\dagger}
\end{equation}

An orthogonal matrix (the corresponding real matrix) has the properties:

\begin{equation}
U^T = U^{-1}
\end{equation}

Where $U^{\dagger}$ is for matrices in Hilbert space (complex matrices). It represents the transpose of the complex conjugate, i.e. $U^{\dagger} = (U^*)^T$.
\end{flushleft}

\begin{flushleft}
Assume we have an orthogonal basis for the $n$-dimensional space $v_i$, where $v_j^T v_i = \delta_{ij}$:

\begin{equation*}
\textbf{v}_i = \begin{bmatrix}
v_{i1}\\
...\\
...\\
v_{in}\\
\end{bmatrix}
\text{ , }
U \textbf{v}_i = \textbf{w}_i
\end{equation*} 

Under orthogonal transformations, both the dot product and orthogonality of vectors is conserved:

$$
\textbf{w}_i^T \textbf{w}_j = (U \textbf{v}_i)^T (U \textbf{v}_j) = \textbf{v}_i^T U^T U \textbf{v}_j = \textbf{v}_i^T U^{-1} U \textbf{v}_j = \textbf{v}_i^T \textbf{v}_j = \delta_{ij}
$$
\end{flushleft}

\subsection*{Jacobi's rotation method (b)}

\begin{flushleft}
We want to solve the eigenvalue problem for a tridiagonal matrix:

\begin{equation}
\begin{pmatrix}
d_0 & e_0 & 0 & ... & 0 & 0 \\
e_1 & d_1 & e_1 & ... & 0 & 0\\
0 & e_2 & d_2 & e_2 & ... & 0\\
\vdots & & & \ddots &  & 0\\
0 & ... & ... & 0 & e_N & d_N\\
\end{pmatrix}
\begin{pmatrix}
u_0\\
u_1\\
\vdots\\
u_{N-1}\\
u_N\\
\end{pmatrix}
= 
\lambda
\begin{pmatrix}
u_0\\
u_1\\
\vdots\\
u_{N-1}\\
u_N\\
\end{pmatrix}
\end{equation}

We want to use Jacobi's rotation method. We define an orthogonal matrix $S$ to operate on our matrix $A$:

\begin{equation}
B = S^T A S
\end{equation}

With all elements in $S$ equal to zero, except:

\begin{equation*}
s_{kk} = s_{ll} = \cos \theta \text{ , } s_{kl} = - s_{lk} = s \sin \theta \text{ , } s_{ii} = 1 \text{ ; } i \neq k \text{ } i \neq l
\end{equation*}

\end{flushleft}

\begin{figure}[ht]\label{Jacobi rotation algorithm}
\begin{framed}
\begin{minipage}[b]{0.45\linewidth}

\begin{flushleft}
$\boldsymbol{1.}$ Find $\max (a_{kl})$.
\end{flushleft}

\begin{flushleft}
$\boldsymbol{2.}$ Compute the rotation parameters $t = \tan \theta$, $c = \cos \theta$, $s = \sin \theta$:
\begin{equation}
\tau = \cot \theta = \frac{a_{ll} - a_{kk}}{2 a_{kl}}
\end{equation}
\begin{equation}
t = - \tau \pm \sqrt{1 + \tau^2}
\end{equation}
\begin{equation}
c = \frac{1}{1 + t^2} \text{ , } s = tc
\end{equation}
\end{flushleft}

\end{minipage}
\hspace{0.5cm}
\begin{minipage}[b]{0.45\linewidth}

\centering

\begin{flushleft}
$\boldsymbol{3.}$ Calculate the elements of $B$:

\begin{equation*}
b_{ii} = a_{ii} \text{ , } i \neq k \text{ } i \neq l
\end{equation*}
\begin{equation*}
b_{ik} = a_{ik} c - a_{il} s \text{ , } i \neq k \text{ } i \neq l
\end{equation*}
\begin{equation*}
b_{il} = a_{il} c + a_{ik} s \text{ , } i \neq k \text{ } i \neq l
\end{equation*}
\begin{equation*}
b_{kk} = a_{kk} c^2 - 2 a_{kl} c*s + a_{ll} s^2
\end{equation*}
\begin{equation*}
b_{ll} = a_{ll} c^2 + 2 a_{kl} c*s + a_{kk} s^2
\end{equation*}
\begin{equation*}
b_{kl} = (a_{kk} - a_{ll}) c*s + a_{kl}(c^2 - s^2)
\end{equation*}
\end{flushleft}

\begin{flushleft}
$\boldsymbol{4.}$ If $\max (b_{kl}) \geq \epsilon$ run the algorithm again.
\end{flushleft}

\end{minipage}
\end{framed}
\caption{Jacobi rotation algorithm for tridiagonal matrix}
\end{figure}

\section*{Benchmarks}

\subsection*{Jacobi rotation}

\begin{flushleft}
Check the Jacobi function, which finds eigenvalues of a matrix, on the matrix $A$:

\begin{equation*}
A = 
\begin{pmatrix}
1 & 2 & 0\\
2 & 1 & 0\\
0 & 0 & 1\\
\end{pmatrix} \text{ , }
\det (A - \lambda \mathbb{1} ) = 
\begin{vmatrix}
1 - \lambda & 2 & 0\\
2 & 1 - \lambda & 0\\
0 & 0 & 1- \lambda\\
\end{vmatrix}
\end{equation*}

$$
\det (A - \lambda \mathbb{1} ) =
(1-\lambda)^3 - 2 \cdot 2(1 - \lambda) = (1 - \lambda)[(1 - \lambda)^2 - 4]
$$

$$
= (1 - \lambda)[\lambda^2 - 2\lambda - 3] \rightarrow \lambda = 1 \cap \lambda = 2
$$

$$
\rightarrow (\lambda - 1)(\lambda - 1)(\lambda - 2)
$$

\end{flushleft}

\end{document}