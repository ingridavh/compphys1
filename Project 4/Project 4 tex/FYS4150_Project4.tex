\documentclass[11pt]{article}
\usepackage{graphicx}
\usepackage[utf8]{inputenc} 
\usepackage{epstopdf}
\usepackage[makeroom]{cancel}
\usepackage{framed}
\usepackage{cite}
\usepackage{hyperref}
\usepackage{amsmath}
\usepackage{amsfonts}
\usepackage{bbold}
\usepackage{braket}

\begin{document}
\title{FYS4150: Project 4}
\author{Ingrid A V Holm}
\maketitle


\section{Introduction}

\begin{flushleft}
The aim of this project is to study the Ising model in two dimensions.
\end{flushleft}

\section{The Ising model (ch. 13)}

\begin{flushleft}
The Ising model describes pahse transitions in two dimensions. At a given critical temperature the model exhibits a phase transition from one magnetic moment (spin) to a phase with zero magnetization. Without an externally applied field, this is described as

\begin{equation}
E = -J \sum^N_{<kl>} s_k s_l
\end{equation}

where $s_k = \pm 1$, $N$ is the total number of spins and $J$ is a coupling constant.
\end{flushleft}

\section{The Metropolis algorithm (ch. 12)}

\begin{flushleft}

\end{flushleft}


\section{2x2 lattice}

\begin{flushleft}
We consider a $2 \times 2$-lattice, meaning $L = 2$. The spins have 16 conifgurations. The probability distribution is given by

\begin{equation}
P_i(\beta) = \frac{e^{- \beta E_i}}{Z}
\end{equation}

where $\beta = \frac{1}{kT}$ is the inverse temperature, $k$ is te Boltzmann constant, $E_i$ is the energy of a state $i$. The partition function is 

\begin{equation}\label{Partition function}
Z = \sum_{i=1}^M e^{- \beta E_i}
\end{equation}

\begin{figure}
\centering
\begin{tabular}{|c|l|l|}
\hline
Number of spins pointing up & Degeneracy & Energy\\
\hline
0 & 1 &  -8J\\
1 & 4 & 0\\
2 & 6 & 8J\\
3 & 4 & 0\\
4 & 1 & -8J\\
\hline
\end{tabular}
\caption{Possible energies for a $2 \times 2$-lattice of spin particles.}
\end{figure}

\end{flushleft}

\begin{flushleft}

This fuction is difficult to compute since we need all states. The Metropolis algorithm only considers \textit{ratios} between probabilities, so we don't need to calculate this (luckily!)

\begin{align*}
\text{Metropolis } \rightarrow  &\frac{P_s}{P_t} = \frac{e^{-(\beta E_s)}}{e^{-(\beta E_k)}}
= e^{-\beta(E_s - E_k)}
= e^{-\beta \Delta E} 
\end{align*}


\end{flushleft}

\begin{flushleft}

The partition function in our case is

\begin{align*}
Z(\beta) &= \sum_{i=1}^M e^{-(\beta E_i)}\\
&= 2 e^{-\beta \cdot (-8J)} + 8 e^{-\beta \cdot 0} + 6 e^{- \beta \cdot 8J}\\
&= 8 + 2e^{8 \beta J} + 6 e^{-8 \beta J}\\
\end{align*}


The energy expectation value

\begin{align*}
\braket{E} & = - \frac{\partial \ln Z}{\partial \beta}= - \frac{\partial \ln Z}{\partial Z} \frac{\partial Z}{\partial \beta} = - \frac{1}{Z} \frac{\partial Z}{\partial \beta}\\
& = - \frac{1}{8 + 2e^{8 \beta J} + 6 e^{-8 \beta J}} ( 8J \cdot 2e^{8 \beta J} - 8 J \cdot 6 e^{-8 \beta J})\\
&= 8J \frac{e^{8 \beta J} - 3 e^{-8 \beta J}}{4 + e^{8 \beta J} + 3 e^{-8 \beta J}}
\end{align*}

with the corresponding variance

\begin{align*}
\sigma^2 = \braket{E^2} - \braket{E}^2 = \frac{1}{Z} \sum_{i=1}^M E_i^2 e^{- \beta E_i} - \Big(
\frac{1}{Z} \sum_{i=1}^M E_i e^{- \beta E_i} 
\Big)^2
\end{align*}

\begin{align*}
\sigma^2 &= 64J^2 \frac{e^{8 \beta J} - 3 e^{-8 \beta J}}{4 + e^{8 \beta J} + 3 e^{-8 \beta J}}
- \big( 8J \frac{e^{8 \beta J} - 3 e^{-8 \beta J}}{4 + e^{8 \beta J} + 3 e^{-8 \beta J}}
\big)^2\\
&= 64J^2 \frac{e^{8 \beta J} - 3 e^{-8 \beta J}}{4 + e^{8 \beta J} + 3 e^{-8 \beta J}}
- 64J^2\big(
\frac{e^{16 \beta J} - 6 \cdot e^{8 \beta J}e^{-8 \beta J} + 9 e^{-16 \beta J}}{(4 + e^{8 \beta J} + 3 e^{-8 \beta J})^2}
\big)\\
&= 
\frac{64J^2}{(4 + e^{8 \beta J} + 3 e^{-8 \beta J})^2}
\Big[
(e^{8 \beta J} - 3 e^{-8 \beta J})(4 + e^{8 \beta J} + 3 e^{-8 \beta J})
-e^{16 \beta J} - 6  + 9 e^{-16 \beta J}
\Big]\\
&= \frac{64J^2}{(4 + e^{8 \beta J} + 3 e^{-8 \beta J})^2}
\Big[
4 e^{8 \beta J} 
- 12 e^{-8 \beta J}  - 6 
\Big] \\
\end{align*}

The specific heat capacity 

\begin{equation}
C_V = \frac{1}{k_B T^2} (\braket{E^2} - \braket{E}^2)
\end{equation}

\begin{align*}
C_V &= \frac{64J^2}{k_B T^2}
\frac{4 e^{8 \beta J} 
- 12 e^{-8 \beta J}  - 6 }{(4 + e^{8 \beta J} + 3 e^{-8 \beta J})^2}\\
\end{align*}


The mean absolute value of the magnetic moment (mean magnetization)

\begin{equation}
\braket{\mathcal{M}} = \frac{1}{Z} \sum_i^M \mathcal{M}_i e^{- \beta E_i}
\end{equation}

\begin{align*}
\braket{\mathcal{M}} = \frac{1}{Z} (-4 e^{8 \beta J} - 2 \cdot 4 e^0 + 0 \cdot e^{-8 \beta J} + 2 \cdot 4 e^0 + 4 e^{8 \beta J}) = 0
\end{align*}



The susceptibility

\begin{equation}
\chi = \frac{1}{k_B T} (\braket{\mathcal{M}^2} - \braket{\mathcal{M}}^2)
\end{equation}

\begin{align*}
\chi &= \frac{1}{k_B T} \frac{1}{Z} \sum_i^M \mathcal{M}_i^2 e^{- \beta E_i}\\
&= \frac{1}{k_B T} 
\frac{1}{Z} (16e^{8 \beta J} + 4 \cdot 4 e^0 + 0 \cdot e^{-8 \beta J} + 4 \cdot 4 e^0 + 16 e^{8 \beta J})\\
&= \frac{16}{k_B T} 
\frac{e^{8 \beta J} + 1}{4 + e^{8 \beta J} + 3 e^{-8 \beta J}} \\
\end{align*}

\end{flushleft}

\begin{flushleft}
The analytical values are found in table (\ref{fig: analytical}).

\begin{figure}[h]
\label{fig: analytical}
\centering
\begin{tabular}{|c|c|c|c|c|}
\hline
$Z$ &$\braket{E}$ & $\braket{\mathcal{M}}$ & $C_V$ & $\chi$\\
\hline
5969.92 & 7.99 & 0 & 0.0856 & 0.00268\\
\hline
\end{tabular}
\caption{Analytical values of the expectation values. Here $J=1$, $k=1$, $T=1.0$ K.}
\end{figure}
\end{flushleft}


\section{Sources}

\end{document}