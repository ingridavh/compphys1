\documentclass[11pt]{article}
\usepackage{graphicx}
\usepackage[utf8]{inputenc} 
\usepackage{epstopdf}
\usepackage[makeroom]{cancel}
\usepackage{framed}
\usepackage{cite}
\usepackage{hyperref}
\usepackage{amsmath}
\usepackage{amsfonts}
\usepackage{bbold}
\usepackage{braket}
\usepackage{textcomp}
\usepackage{subcaption}

\begin{document}
\title{FYS4150: Project 4}
\author{Ingrid A V Holm}
\maketitle


\section{Introduction}

\begin{flushleft}
The aim of this project is to study the Ising model in two dimensions.
\end{flushleft}

\section{The Ising model (ch. 13)}

\begin{flushleft}
The Ising model describes pahse transitions in two dimensions. At a given critical temperature the model exhibits a phase transition from one magnetic moment (spin) to a phase with zero magnetization. Without an externally applied field, this is described as

\begin{equation}
E = -J \sum^N_{<kl>} s_k s_l
\end{equation}

where $s_k = \pm 1$, $N$ is the total number of spins and $J$ is a coupling constant.
\end{flushleft}

\section{The Metropolis algorithm (ch. 12)}

\begin{flushleft}

\end{flushleft}


\section{2x2 lattice}

\begin{flushleft}
We consider a $2 \times 2$-lattice, meaning $L = 2$. The spins have 16 conifgurations. The probability distribution is given by

\begin{equation}
P_i(\beta) = \frac{e^{- \beta E_i}}{Z}
\end{equation}

where $\beta = \frac{1}{kT}$ is the inverse temperature, $k$ is te Boltzmann constant, $E_i$ is the energy of a state $i$. The partition function is 

\begin{equation}\label{Partition function}
Z = \sum_{i=1}^M e^{- \beta E_i}
\end{equation}

\begin{figure}
\centering
\begin{subfigure}{.4\textwidth}
  \centering
  \includegraphics[width=\linewidth]{spins_2by2.png}
  \caption{Different spin configurations}
  \label{fig:sub1}
\end{subfigure}%
\begin{subfigure}{.6\textwidth}
  \centering
  \begin{tabular}{|c|l|l|l|}
\hline
\textnumero of spins & Degeneracy & E& M\\
 pointing up  &&&\\
\hline
0 & 1 &  -8J & -4\\
1 & 4 & 0 & -2\\
2 & 4 & 0 & 0\\
2 & 2 & 8J & 0\\
3 & 4 & 0 & 2\\
4 & 1 & -8J & 4\\
\hline
\end{tabular}
\caption{Possible energies for a $2 \times 2$-lattice of spin particles.}
  \label{fig::spins 2 by 2}
\end{subfigure}
\caption{Spins for a $2 \times 2$-lattice with periodic boundary conditions.}
\label{fig:BPM no steering}
\end{figure}



\end{flushleft}

\begin{flushleft}

This fuction is difficult to compute since we need all states. The Metropolis algorithm only considers \textit{ratios} between probabilities, so we don't need to calculate this (luckily!)

\begin{align*}
\text{Metropolis } \rightarrow  &\frac{P_s}{P_t} = \frac{e^{-(\beta E_s)}}{e^{-(\beta E_k)}}
= e^{-\beta(E_s - E_k)}
= e^{-\beta \Delta E} 
\end{align*}


\end{flushleft}

\begin{flushleft}

The partition function in our case is

\begin{align*}
Z(\beta) &= \sum_{i=1}^M e^{-(\beta E_i)}\\
&= 2 e^{8J \beta} + 2 e^{-8J \beta} + 12\\
&= 4(3 + \cosh (8 J \beta))\\
\end{align*}


The energy expectation value

\begin{align*}
\braket{E} & = - \frac{\partial \ln Z}{\partial \beta}= - \frac{\partial \ln Z}{\partial Z} \frac{\partial Z}{\partial \beta} = - \frac{1}{Z} \frac{\partial Z}{\partial \beta}\\
& = - \frac{4 \cdot 8J \sinh(8J \beta)}{4(3 + \cosh (8 J \beta))}\\
&= - 8J \frac{\sinh(8J \beta)}{3 + \cosh(8J \beta)}  
\end{align*}

with the corresponding variance

\begin{align*}
\sigma^2 &= \braket{E^2} - \braket{E}^2 = \frac{1}{Z} \sum_{i=1}^M E_i^2 e^{- \beta E_i} - \Big(
\frac{1}{Z} \sum_{i=1}^M E_i e^{- \beta E_i} 
\Big)^2\\
 &= \frac{8^2 J^2  2 e^{8J \beta} + 8^2 J^2 2 e^{-8J \beta} + 0 \cdot 12}{ 4(3 + \cosh (8 J \beta))}
- \Big(  - 8J \frac{\sinh(8J \beta)}{3 + \cosh(8J \beta)}  \Big)^2\\
&= 32J^2 \frac{ 2 \cosh (8J \beta)}{ (3 + \cosh (8 J \beta))}
- 64 J^2 \frac{\sinh^2(8J \beta)}{(3 + \cosh(8J \beta))^2}\\
&= J^264 \frac{\cosh (8J \beta)(3 + \cosh (8 J \beta)) - \sinh^2(8J \beta)}{ (3 + \cosh (8 J \beta))^2}\\
&= J^264 \frac{
3\cosh (8J \beta) + \cosh^2 (8 J \beta)
 - \sinh^2(8J \beta)}{ (3 + \cosh (8 J \beta))^2}\\
 &= J^264 \frac{
3\cosh (8J \beta) + 1}{ (3 + \cosh (8 J \beta))^2}\\
\end{align*}

The specific heat capacity 

\begin{equation}
C_V = \frac{1}{k_B T^2} (\braket{E^2} - \braket{E}^2)
\end{equation}

\begin{align*}
C_V &= \frac{64J^2}{k_B T^2}
\frac{3\cosh (8J \beta) + 1}{ (3 + \cosh (8 J \beta))^2}
\end{align*}


The mean absolute value of the magnetic moment (mean magnetization)

\begin{align*}
\braket{|\mathcal{M}|} 
&= \frac{1}{Z} \sum_i^M |\mathcal{M}_i| e^{- \beta E_i}\\
&= \frac{1}{4(3 + \cosh (8 J \beta))} (4e^{8J \beta} + 4 \cdot 2 e^0 + 2 \cdot 4 e^0 +4 e^{8 J \beta})\\
& =2 \frac{e^{8J \beta} + 2 }{3 + \cosh (8 J \beta)} \\
\end{align*}



The susceptibility

\begin{equation}
\chi = \frac{1}{k_B T} (\braket{\mathcal{M}^2} - \braket{\mathcal{M}}^2)
\end{equation}

\begin{align*}
\chi &= \frac{1}{k_B T} (\frac{1}{Z} \sum_i^M \mathcal{M}_i^2 e^{- \beta E_i}- \Big(\frac{1}{Z} \sum_i^M \mathcal{M}_i e^{- \beta E_i}\Big)^2)\\
&= \frac{1}{k_B T} 
\frac{(16e^{8J \beta} + 4 \cdot 4 e^0 + 4 \cdot 4 e^0 + 16 e^{8J \beta})}{4(3 + \cosh (8 J \beta))} \\
&= \frac{1}{k_B T} 
\frac{8e^{8J \beta} +8}{(3 + \cosh (8 J \beta))} \\
\end{align*}

\end{flushleft}

\begin{flushleft}
The analytical values (calculated in Python) are found in table (\ref{fig:: analytical}).

\begin{figure}[h]
\label{fig:: analytical}
\centering
\begin{tabular}{|c|c|c|c|c|}
\hline
$Z$ &$\braket{E}$ & $\braket{|\mathcal{M}|}$ & $C_V$ & $\chi$\\
\hline
$5973.917$ & $-7.9839$ & $3.9946$ & $0.1283$ & $15.9732$\\
\hline
\end{tabular}
\caption{Analytical values of the expectation values. Here $J=1$, $k=1$, $T=1.0$ K.}
\end{figure}
\end{flushleft}

\begin{flushleft}
For $mcs = 1000000$ cycles the results are very similar to the analytic results. 
\end{flushleft}

\section{Sources}

\end{document}