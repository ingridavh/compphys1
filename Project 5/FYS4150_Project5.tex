\documentclass[11pt]{article}
\usepackage{graphicx}
\usepackage[utf8]{inputenc} 
\usepackage{epstopdf}
\usepackage[makeroom]{cancel}
\usepackage{framed}
\usepackage{cite}
\usepackage{hyperref}
\usepackage{amsmath}
\usepackage{amsfonts}
\usepackage{bbold}
\usepackage{braket}
\usepackage{cite}
\usepackage{textcomp}
\usepackage{subcaption}
\usepackage{float}

\usepackage{geometry}
\geometry{legalpaper, margin=1.2in}

\begin{document}
\title{FYS4150: Project 5}
\author{Ingrid A V Holm}
\maketitle

\section{Physical background}

\subsection{Quantum dots}

\begin{flushleft}
Consider a system of electrons confined in a pure three-dimensional isotropic harmonic oscillator potential, with an idealized Hamiltonian given by
\begin{equation}
\hat{H} = \sum_{i=1}^N \big(-\frac{1}{2} \nabla_i^2 + \frac{1}{2} \omega^2 r^2_i  \big) + \sum_{i<j} \frac{1}{r_{ij}},
\end{equation}

where we have used natural units $\hbar = c = e=m_e=1$, and all energies are in atomic units a.u. Our system consists of $N=2$ particles, and the hamiltonian describes a harmonic oscillator and the repulsive interaction between two electrons
\begin{equation}
\hat{H}_1=\sum_{i<j}\frac{1}{r_{ij}},
\end{equation}
where the distance between electrons is given by $r_{ij}= \sqrt{\textbf{r}_1-\textbf{r}_2}$. The modulus for the position of a single electron is given as $r_i = \sqrt{x_i^2+y_i^2+z_i^2}$.
\end{flushleft}

\subsection{The non-interacting case}
\begin{flushleft}
For the unperturbed system the Hamiltonian is
\begin{equation}
\hat{H}_0 = \sum_{i=1}^N \big( -\frac{1}{2} \nabla_i^2 + \frac{1}{2} \omega^2 r_i^2 \big).
\end{equation}
If we set $\hbar \omega=1$ the exact energy for 2 electrons is 3 a.u. The \textit{wavefunction} for an electron in an oscillator potential in 3D is
\begin{equation}
\phi_{n_x,n_y,n_z}(x,y,z) = AH_{n_x} (\sqrt{\omega}x)H_{n_y}(\sqrt{\omega}y)H_{n_z}(\sqrt{\omega}z) \exp(- \omega(x^2+y^2+z^2)/2)),
\end{equation}
where the functions $H_{n_x}(\sqrt{\omega}x)$ are the hermite polynomials, and $A$ is a normalization constant. For the ground state $n_i=0$ and the energy of a single electron is $\epsilon_{n_x, n_y, n_z}= \omega(n_x+n_y+n_z + 3/2)=3/2 \omega$. In this case the electrons don't interact, so the total energy is just the sum of energies
\begin{equation*}
E_0 = \frac{3\omega}{2} + \frac{3 \omega}{2} = 3 \omega
\end{equation*}
\end{flushleft}

\begin{flushleft}
The total spin should be $0$. Electrons are fermions, so they must have antisymmetric wavefunctions. If their quantum numbers are all the same $n_i=0$, they must have different spin, so $S_{total} = \frac{1}{2}-\frac{1}{2}=0$.
\end{flushleft}

\pagebreak

\subsection{Trial wave functions}

\begin{flushleft}
The trial wavefunctions we want to use are
\begin{equation}
\Psi_{T1} (\textbf{r}_1, \textbf{r}_2) = C \exp (- \alpha \omega (r_1^2 + r_2^2)/2),
\end{equation}
\begin{equation}
\Psi_{T2}(\textbf{r}_1, \textbf{r}_2) = C \exp (- \alpha \omega (r_1^2+r_2^2)/2) \exp \big( \frac{r_{12}}{2(1+\beta r_{12})} \big),
\end{equation}
where $\alpha, \beta$ are variational parameters. To find the energy of the first trial state, we use the hamiltonian operator on the state
\begin{align*}
\hat{H}_0 \Psi_{T1} &=\frac{1}{2} \big(-(\nabla_1^2 + \nabla_2^2) + \omega^2 r_1^2 + \omega r_2^2 \big) 
C \exp (- \alpha \omega (r_1^2 + r_2^2)/2)\\
&= \frac{1}{2} (\alpha \omega(\nabla_1 + \nabla_2)(x_1 + y_1 + z_1 + x_2 + y_2 + z_2) + \omega^2(r_1^2 + r_2^2))C \exp (- \alpha \omega (r_1^2 + r_2^2)/2)\\
&= \frac{1}{2} (6 \alpha \omega + -\alpha^2 \omega^2 (r_1^2 + r_2^2) ) + \omega^2(r_1^2 + r_2^2))C \exp (- \alpha \omega (r_1^2 + r_2^2)/2)\\
&= 3 \alpha \omega - \frac{\omega^2}{2}(1-\alpha^2) C \exp (- \alpha \omega (r_1^2 + r_2^2)/2)\\
&= E_{T1} \Psi_{T1}\\
\end{align*}
where we have used that $\nabla x_1 e^{- \alpha \omega (r_1^2+r_2^2)/2}= (1 - \alpha \omega x_1^2)e^{- \alpha \omega (r_1^2+r_2^2)/2}$. So the energy of the first test function is 
\begin{equation}
E_{T1} = 3 \alpha \omega + \frac{\omega^2}{2}(1 - \alpha^2)(r_1^2 + r_2^2) + 3 \alpha \omega.
\end{equation}
\end{flushleft}

\subsection{Hydrogen atom}
\begin{flushleft}
We can view the hydrogen atom as a harmonic oscillator.

\end{flushleft}

\section{Implementation}
\subsection{Variational Monte Carlo}





\end{document}